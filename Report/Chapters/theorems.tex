\section{Definitions and Theorems for Mathematical Models}
\label{ch:theorems}

\subsection{Recurrence equation}

This is our general linear recurrence equation with constant coefficients:
\begin{equation}\label{eq:an}
    x_{n+1} = a_0x_n + a_1x_{n-1} + a_2x_{n-2} + \dots + a_{q}x_{n-q}
\end{equation}

The characteristic polynomial of \eqref{eq:an}
\begin{equation}\label{eq:fchar}
    f(\lambda) = \lambda^{q+1} - a_0\lambda^q - a_1\lambda^{q-1} - a_2\lambda^{q-2} - \dots - a_{q-1}\lambda - a_q.
\end{equation}

\begin{definition}
A root $\lambda$ of $f$ with the maximal absolute value $|\lambda|$ will be referred to as a leading root of the general linear recurrence relation \eqref{eq:an}.
\end{definition}

\begin{definition}
The geormetric mean of $n$ values $x_1,\dots,x_n$ is
\begin{equation}\label{eq:geometricmean}
   \left(\prod\limits_{i=1}^n x_i \right)^{\frac1n} = \sqrt[n]{x_1 x_2 \cdots x_n}
\end{equation}
\end{definition}

\subsection{Results}

\begin{theorem}\label{thm:lim}
Let $a_k \geq 0$ for all $k \in \{0,\dots,q\}$ and $a_{k_0} > 0$ for some $k_0\in \{0,\dots,q\}$.
\begin{enumerate}[(a)]
    \item (Cauchy, 1829) The polynomial $f(\lambda)$ from \eqref{eq:fchar} has exactly one positive real root $r$. Besides, the root $r$ is simple and, for any other root $\lambda \in \bC$, we have $|\lambda| < r$. Consequently, $r$ is the leading root of \eqref{eq:an}.
    \item For any positive solution $x_n$ of \eqref{eq:an}, there exists $C > 0$ such that
    \begin{equation}\label{eq:crn}
        x_n \sim Cr^n \text{ as } n\to\infty.
    \end{equation}
    It follows from \eqref{eq:crn} that if $r < 1$ then the epidemic fades away, whereas if $r > 1$ then it will spread indefinitely.
\end{enumerate}
\begin{pf}
\begin{enumerate}[(a)]
\item The equation $f(\lambda) = 0$ is equivalent to

$\begin{aligned}
0 &= \lambda^{q+1} - a_0\lambda^q - a_1\lambda^{q-1} - a_2\lambda^{q-2} - \dots - a_{q-1}\lambda - a_q \\
\text{dividing across by } \lambda^{q+1} \\
  &= 1 - \frac{a_0}{\lambda} - \frac{a_1}{\lambda^2} - \frac{a_2}{\lambda^3}-\dots - \frac{a_{q-1}}{\lambda^q} -\frac{a_q}{\lambda^{q+1}}
\end{aligned}$
    
And so
\begin{equation} \label{eq:glambda}
1 =\underbrace{\frac{a_0}{\lambda} + \frac{a_1}{\lambda^2} + \frac{a_2}{\lambda^3}+\dots + \frac{a_{q-1}}{\lambda^q} +\frac{a_q}{\lambda^{q+1}}}_{g(\lambda)}    
\end{equation}

Since $a_{k_0} > 0$ for some $k_0$, and the remaining $a_k$ are non-negative, $g(\lambda)$ is strictly monotone decreasing in $\lambda > 0$ 
%(if $c\lambda$ is increasing,then $\frac{c}\lambda$ is decreasing)
,  and we have the limits

\begin{itemize}
    \item $\lim\limits_{\lambda\to0^+}g(\lambda)=+\infty$
    \item $\lim\limits_{\lambda\to+\infty}g(\lambda)=0^+$
\end{itemize}

Hence, there is exactly one positive value $\lambda = r$ that satisfies this $g(r) = 1$, that is,

$$1 =\frac{a_0}{r} + \frac{a_1}{r^2} + \frac{a_2}{r^3}+\dots + \frac{a_{q-1}}{r^q} +\frac{a_q}{r^{q+1}}.$$

Now, let $\lambda\in \bC \bs \{0\}$ be another root of $f$. We obtain from \eqref{eq:glambda} (using the triangle inequality) that

$$1\leq \frac{a_0}{|\lambda|} + \frac{a_1}{|\lambda|^2} + \frac{a_2}{|\lambda|^3}+\dots + \frac{a_{q-1}}{|\lambda|^q} +\frac{a_q}{|\lambda|^{q+1}}$$

Since $g$ is strictly decreasing,

$g(r)\leq g(|\lambda|)$ which implies $|\lambda|\leq r$.

We next need to show that the root $r$ is simple. Denote by $r'$ the largest non-negative root of the
derivative $f'(\lambda)$ that exists for the following reason. If $a_k > 0$ for some $k < q$ then the
polynomial $\frac{1}{q+1} f'(\lambda)$ satisfies the hypotheses of the present theorem and, by the above
argument, $f'(\lambda)$ has exactly one positive root, that is $r'$. If $a_k = 0$ for all $k < q$ then
$f'(\lambda) = (q + 1) \lambda^q$ has the only root $0$, and, hence, $r' = 0$. 

Let us verify that $r' < r$, which will also imply that $r$ is simple. If $r' = 0$ then it is clear. If $r' > 0$ then it follows
from $f'(r') = 0$ that

$\begin{aligned}
f'(\lambda) &=(q+1)\lambda^q -qa_0\lambda^{q-1} -(q-1)a_1\lambda^{q-2} -(q-2)a_2\lambda^{q-3} -\cdots - a_{q-1}-0\\
\frac{1}{q+1}f'(\lambda) &=\lambda^q -\frac{q}{q+1}a_0\lambda^{q-1} -\frac{q-1}{q+1}a_1\lambda^{q-2} -\frac{q-2}{q+1}a_2\lambda^{q-3} -\cdots - \frac{1}{q+1}a_{q-1}\\
\frac{1}{q+1}f'(r') &=(r')^q -\frac{q}{q+1}a_0(r')^{q-1} -\frac{q-1}{q+1}a_1(r')^{q-2} -\frac{q-2}{q+1}a_2(r')^{q-3} -\cdots - \frac{1}{q+1}a_{q-1}\\
0 &=(r')^q -\frac{q}{q+1}a_0(r')^{q-1} -\frac{q-1}{q+1}a_1(r')^{q-2} -\frac{q-2}{q+1}a_2(r')^{q-3} -\cdots - \frac{1}{q+1}a_{q-1}\\
(r')^q &=\frac{q}{q+1}a_0(r')^{q-1} +\frac{q-1}{q+1}a_1(r')^{q-2} +\frac{q-2}{q+1}a_2(r')^{q-3} +\cdots + \frac{1}{q+1}a_{q-1}\\
& \text{dividing both sides by } (r')^q>0 \\
1 &= \frac{qa_0}{(q+1)r'} +\frac{(q-1)a_1}{(q+1)(r')^2} +\dots+\frac{a_{q-1}}{(q+1)(r')^q} \\
  &= \rbr{\frac{q+1-1}{q+1}}\frac{a_0}{r'} +
     \rbr{\frac{q+1-2}{q+1}}\frac{a_1}{(r')^2} +\dots+
     \rbr{\frac{q+1-q}{q+1}}\frac{a_{q-1}}{(r')^q}\\
  &= \rbr{1-\frac{1}{q+1}}\frac{a_0}{r'} +
     \rbr{1-\frac{2}{q+1}}\frac{a_1}{(r')^2} +\dots+
     \rbr{1-\frac{q}{q+1}}\frac{a_{q-1}}{(r')^q}\\
  &< \frac{a_0}{r'} +\frac{a_1}{(r')^2} +\dots+\frac{a_{q-1}}{(r')^q}
\end{aligned}$

So $g(r')>1$, but $g(r)=1$

Again, since the function $g$ is strictly decreasing, $g(r')>g(r)\imp r'<r$ 

Therefore $r$  is simple, and is the leading root of  \eqref{eq:an}.

\item Let $\lambda_1, \lambda_2, \dots$ be all other distinct roots of $f$ apart from $r$ (so that $\lambda_k$ are negative or imaginary). Any solution $x_n$ of \eqref{eq:an} has the form

\begin{equation} \label{eq:Crnxn}
x_n = Cr^n +\tilde{x}_n    
\end{equation}

where $\tilde{x}_n$ is a linear combination of the functions $n^j\lambda^n_k$ . Since by (a) we have $|\lambda_k| < r$, it follows that

\begin{equation} \label{eq:xnorn}
|\tilde{x}_n| = o(r^n) \ \text{as} \ n\to\infty    
\end{equation}

Since $x_n > 0$, it follows from \eqref{eq:Crnxn} and \eqref{eq:xnorn} that $C \geq 0$. 

We need to show that $C > 0$.

Consider a new sequence 

$$ X_n = \frac{x_n}{r^n}.$$

This satisfies the equation

\begin{equation} \label{eq:AkXk}
X_{n+1}=A_0X_n+A_1X_{n-1}+\dots+A_qX_{n-q}
\end{equation}

with $A_k = \frac{a_k}{r^{k+1}}$. 

Since

$\begin{aligned}
A_0X_0+A_1X_{n-1}+\dots+A_qX_{n-q} 
&= \frac{a_0}{r^1}\frac{x_n}{r^n} + \frac{a_1}{r^2}\frac{x_{n-1}}{r^{n-1}} +  \dots + \frac{a_q}{r^{q+1}}\frac{x_{n-q}}{r^{n-q}}\\
&= \frac{a_0x_n}{r^{n+1}} + \frac{a_1x_{n-1}}{r^{n+1}} + \dots + \frac{a_qx_{n-q}}{r^{n+1}}\\
&= \frac1{r^{n+1}}\rbr{a_0x_n+a_1x_{n-1}+\dots + a_qx_{n-q}}\\
&= \frac1{r^{n+1}}\rbr{x_{n+1}} \quad \text{from } \eqref{eq:an} \\
&= \frac{x_{n+1}}{r^{n+1}}\rbr{x_{n+1}} \\
&= X_{n+1}
\end{aligned}$

Since $r$ is a root of $f$, we have

\begin{align}
  A_0+A_1+\dots+A_q &= \frac{a_0}{r^{1}}+\frac{a_1}{r^{2}}+\dots+\frac{a_q}{r^{q+1}} \nonumber \\ 
                     &= g(r) \nonumber 
\end{align}

This implies, by \eqref{eq:glambda}, and $g(r)=1$ that 

\begin{equation}
     A_0+A_1+\dots+A_q = 1 \label{eq:A0Aq}
\end{equation}

Set $c := \min(X_1, \dots , X_{q+1}) > 0$ since $x_n$ have positive initial values. Then we obtain from \eqref{eq:AkXk} and \eqref{eq:A0Aq} by induction that
$X_n \geq c$ for all $n \in \bN$, which implies

$x_n\geq cr^n$

But equating with \eqref{eq:Crnxn} we get

\begin{align}
      Cr^n + \tilde{x}_n   &\geq cr^n  \nonumber\\
    C +\frac{\tilde{x}_n}{r^n}    &\geq c \nonumber \\
  \text{Taking the limit as } n\to\infty \nonumber \\
   C \geq c &> 0
\end{align}

Therefore $C>0$, as required.
\end{enumerate}
\end{pf}
\end{theorem}

\begin{theorem}\label{thm:rineq}
Let $a_k \geq 0$ for all $k = 0, \dots , q$. Denote $a = a_1 + \dots + a_q, b = 1 - a_0$
and assume that $a > 0, b > 0$.
\begin{enumerate}[(a)]
    \item We have the equivalences: $r < 1 \iff a < b$ and $r > 1 \iff a > b$.
    \item Let $m \geq 1$ be such that $a_1 = \dots = a_{m-1} = 0$ and $a_m > 0$. Then
    \begin{equation} \label{eq:maxminr}
        \min\rbr{1,\rbr{\frac{a}{b}}^{1/m}}\leq r \leq \max\rbr{1,\rbr{\frac{a}{b}}^{1/m}}
    \end{equation}
\end{enumerate}
\begin{pf}
\begin{enumerate}[(a)]
    \item We have 
    
    $\begin{aligned}
    f(1) &= 1 - a_0 - a1 - \dots - a_q \\
           &= \underbrace{(1-a_0)}_{b}-\underbrace{(a_1+\dots+a_q)}_{a}\\
           &= b - a
    \end{aligned}$
    
    We know $f$ is increasing.
    
     If $r=1$ then the inequality obviously holds.
    
    So if $r<1$, we have $f(1)>0$ and then $b - a>0 \imp a<b$.
    
    And if $r>1$, we have $f(1)<0$ and then $b - a<0 \imp a>b$
    
    \item $f(r)=0$ is equivalent to 
    
    $r^{q+1} - a_0r^q - a_1r^{q-1} - a_2r^{q-2} - \dots - a_{q-1}r - a_q=0$
    
    But any $a_1,\dots,a_{m-1}$ are all zero
    
    \imp $r^{q+1} - a_0r^q - a_mr^{q-m} - a_{m+1}r^{q-m-1} - \dots - a_{q-1}r - a_q=0$ 
     
    \imp $r^{q+1} - (1-b)r^q - a_mr^{q-m} - \dots -  a_q=0$
    
    \imp $r^{q+1} - r^q + br^q - a_1r^{q-m} - \dots -  a_q=0$
    
    \imp $r^{q+1} - r^q =-br^q + a_mr^{q-m} + \dots  + a_q$
    
    If $r>1$ then $r^{q+1}>r^q$ and so $r^{q+1}-r^q>0$
    
    and so 
    
    $0<-br^q + a_mr^{q-m} +  \dots + a_q$
    
    \imp $\begin{aligned}[t]
    br^q 
    &<    a_mr^{q-m} +  \dots + a_q\\
    &\leq a_mr^{q-m} +  \dots + a_qr^{q-m} \\
    &= (a_m + \dots + a_q)r^{q-m} \\
    &= ar^{q-m} 
    \end{aligned}$
    
    So $br^q < ar^{q-m} \iff r^m=\frac{a}{b} \iff r<\rbr{\frac{a}{b}}^{1/m}$
    
    And if $r<1$ we get $r<\rbr{\frac{a}{b}}^{1/m}$.
    
    We can combine both cases with $x\leq \max\rbr{1,x}$ and $x\geq \min\rbr{1,x}$ (for any $x\in \bR$) to get \eqref{eq:maxminr}, as required.
\end{enumerate}
\end{pf}
\end{theorem}

\begin{nremark}
The inequality \eqref{eq:maxminr} will be useful to help approximate the leading root $r$ in \ref{thm:rapprox}.
\end{nremark}

\begin{lemma}
For the model described by recurrence equation \eqref{eq:an} we have

$$R_0=\frac{a}{b}$$

\begin{pf}
Let $u$ be the number of people infected on some day, set this to be day $0$. On the day $k = 1, \dots , q$ the number $c_ku$ of them become ill and can infect other people. On the day $k + 1$ they infect $ac_ku$ people while $bc_ku$ of them get isolated. On the day $k + 1$, the remaining $(1 - b) c_ku$ people infect further $a (1 - b) c_ku$ people. Continuing this way, we obtain that this group of $c_ku$ people infects in total

$\begin{aligned}
ac_ku+a(1-b)c_ku+a(1-b)^2c_ku+\dots 
&= ac_ku\sum\limits_{n=0}^\infty (1-b)^n \\
&= \frac{ac_ku}{1-(1-b)} \quad \text{since} \ 0<1-b<1\\
&=\frac{a}{b}c_ku\\
\end{aligned}$ 

other people.

Hence, the initial group of $u$ people infects in total 

$$\sum\limits_{k=0}^q \frac{a}{b}c_ku 
= \frac{a}{b}u\sum\limits_{k=0}^q c_k
= \frac{a}{b}u$$

other people.

And we defined $R_0$ as being the unit reprodiction number per infected person ($u=1$).

And so we get the result $R_0=\frac{a}{b}$ as required.
\end{pf}
\end{lemma}


We can then apply Newton's method \cite{EncyMathNewton} to find a better approximation for $r$.

\begin{definition}
Let $\cbr{r_n}_{n\geq0}$ be a sequence defined by

\begin{equation}\label{eq:newtoneq}
r_{n+1} = r_n - \frac{f(r_n)}{f'(r_n)}, \quad n\geq0
\end{equation}

Then the limit converges to the leading root $r$, i.e.

\begin{equation}
\cbr{r_n} \to r \quad \text{as } n\to\infty
\end{equation}
\end{definition}

\begin{result}\label{thm:rapprox}
A good approximation for the leading root $r$ is 

\begin{equation} \label{eq:req1}
r\approx  \frac{q\rbr{\frac{a}{b}}^{\frac{1}{2q}} - (q-1)(1-b) + \rbr{ab}^{-\frac12}}{(q+1) - q(1-b)\rbr{\frac{a}{b}}^{-\frac{1}{2q}}}
\end{equation}

\begin{pf}
Our characteristic polynomial (via \eqref{eq:fchar}) for our recurrence equation \eqref{eq:xnrecurr} is

\begin{equation} \label{eq:fchar1}
f(\lambda) = \lambda^{q+1} - (1-b)\lambda^q - a
\end{equation}

Let $r$ be its leading root, i.e. $f(r)=0$ 

Then by \ref{thm:rineq}, we get 

\begin{equation}
 \min\rbr{1,\rbr{\frac{a}{b}}^{1/q}}\leq r \leq \max\rbr{1,\rbr{\frac{a}{b}}^{1/q}}
\end{equation}

since $m=q$ in our polynomial.

Taking the \textit{geometric mean} of the bounds, defined as \eqref{eq:geometricmean}, we get an approximation for $r$

\begin{equation}\label{eq:rapprox0}
r_0 = \rbr{1\cdot \rbr{\frac{a}{b}}^{1/q}}^{\frac12}= \rbr{\frac{a}{b}}^{\frac1{2q}}
\end{equation}

The derivative of our characteristic polynomial \eqref{eq:fchar1} is
\begin{equation} \label{eq:fcharprime1}
f'(\lambda) = (q+1)\lambda^q - q(1-b)\lambda^{q-1}
\end{equation}

Then we apply Newton's method \eqref{eq:newtoneq} once to get

$\begin{aligned}
r_1 
&= r_0 - \frac{f(r_0)}{f'(r_0)} \\
&= r_0 - \frac{r_0^{q+1} - (1-b)r_0^q - a}{(q+1)r_0^q - q(1-b)r_0^{q-1}} \\
&= \rbr{\frac{a}{b}}^{\frac1{2q}} - \frac{\rbr{\rbr{\frac{a}{b}}^{\frac1{2q}}}^{q+1} - (1-b)\rbr{\rbr{\frac{a}{b}}^{\frac1{2q}}}^q - a}{(q+1)\rbr{\rbr{\frac{a}{b}}^{\frac1{2q}}}^q - q(1-b)\rbr{\rbr{\frac{a}{b}}^{\frac1{2q}}}^{q-1}} \\
&= \rbr{\frac{a}{b}}^{\frac1{2q}} - \frac{\rbr{\frac{a}{b}}^{\frac{q+1}{2q}} - (1-b)\rbr{\frac{a}{b}}^{\frac12} - a}{(q+1)\rbr{\frac{a}{b}}^{\frac12} - q(1-b)\rbr{\frac{a}{b}}^{\frac{q-1}{2q}}}\\
&= \rbr{\frac{a}{b}}^{\frac1{2q}} - \frac{\rbr{\frac{a}{b}}^{-\frac12}}{\rbr{\frac{a}{b}}^{-\frac12}} \cdot
\frac{\rbr{\frac{a}{b}}^{\frac{q+1}{2q}} - (1-b)\rbr{\frac{a}{b}}^{\frac12} - a}{(q+1)\rbr{\frac{a}{b}}^{\frac12} - q(1-b)\rbr{\frac{a}{b}}^{\frac{q-1}{2q}}}\\
&= \rbr{\frac{a}{b}}^{\frac1{2q}} - \frac{\rbr{\frac{a}{b}}^{\frac1{2q}} - (1-b) - a\rbr{\frac{a}{b}}^{-\frac12}}{(q+1) - q(1-b)\rbr{\frac{a}{b}}^{-\frac{1}{2q}}}\\
&= \rbr{\frac{a}{b}}^{\frac1{2q}} - \frac{\rbr{\frac{a}{b}}^{\frac1{2q}} - (1-b) - a\rbr{\frac{a}{b}}^{-\frac12}}{(q+1) - q(1-b)\rbr{\frac{a}{b}}^{-\frac{1}{2q}}}\\
&= \frac{(q+1)\rbr{\frac{a}{b}}^{\frac{1}{2q}} - q(1-b)}{(q+1) - q(1-b)\rbr{\frac{a}{b}}^{-\frac{1}{2q}}} - \frac{\rbr{\frac{a}{b}}^{\frac1{2q}} - (1-b) - a\rbr{\frac{a}{b}}^{-\frac12}}{(q+1) - q(1-b)\rbr{\frac{a}{b}}^{-\frac{1}{2q}}}\\
&= \frac{(q+1)\rbr{\frac{a}{b}}^{\frac{1}{2q}} - q(1-b) - \rbr{\frac{a}{b}}^{\frac1{2q}} + (1-b) + a\rbr{\frac{a}{b}}^{-\frac12}}{(q+1) - q(1-b)\rbr{\frac{a}{b}}^{-\frac{1}{2q}}}\\
&= \frac{q\rbr{\frac{a}{b}}^{\frac{1}{2q}} - (q-1)(1-b) + a\rbr{\frac{b}{a}}^{-\frac12}}{(q+1) - q(1-b)\rbr{\frac{a}{b}}^{-\frac{1}{2q}}}\\
&= \frac{q\rbr{\frac{a}{b}}^{\frac{1}{2q}} - (q-1)(1-b) + \rbr{ab}^{-\frac12}}{(q+1) - q(1-b)\rbr{\frac{a}{b}}^{-\frac{1}{2q}}}\\
\end{aligned}$
\end{pf}
\end{result}

While one iteration of Newton's method is often enough with our initial choice $r_0=\rbr{R_0}^{\frac1{2q}}$, we will show that it converges to $r$ in the limit

\begin{lemma}
Let $r'$ be the largest non-negative root of $f'(\lambda)$, the derivative of \eqref{eq:fchar}. 

If we choose an initial $r_0 > r'$ then $\lim\limits_{n\to\infty} r_n = r$.

\begin{pf}
Since $f(\lambda)$ defined above has leading root $r$ and has leading coefficient $1$, we know that $f(\lambda)>0$ for $\lambda>r$.

Also, since $r$ is the only positive root, by part (a) of \ref{thm:lim}, then $f(\lambda)<0$ for $0<\lambda<r$.

We have 

\begin{equation}\label{eq:r1newton}
r_1 = r_0 - \frac{f(r_0)}{f'(r_0)}
\end{equation}

We require $r_0$ and $r_1$ to be in the open interval $(r',\infty)$ in order for $r_n$ to converge to $r$.

There are two cases to consider:

\begin{itemize}
\item  If our chosen $r_0$ is less than $r$, then $f(r_0)<0$ and $f'(r_0)>0$.

$$r_1 = r_0 - \underbrace{\frac{f(r_0)}{f'(r_0)}}_{<0} > r_0 > r'$$


Then $r_1>r'$ and the sequence will converge.

\item   If our chosen $r_0$ is greater than $r$, then $f(r_0)>0$ and $f'(r_0)>0$.

By a similar argument to above, 

$$r_1 < r_0$$ so we need to do more work to show $r_1 > r'$ in this case

Since $f$ is continuous in $[r_1,r_0]$ and differentiable in $(r_1,r_0)$, we can apply the \textit{Mean Value Theorem} to imply that there exists some point $c\in (r_1,r_0)$ such that 

\begin{equation}
f'(c)=\frac{f(r_0)-f(r_1)}{r_0-r_1}
\end{equation}

We can rearrange this equation

$$f(r_1)= f(r_0) + f'(c)(r_1-r_0)$$

But from \eqref{eq:r1newton} we have 

$$r_1-r_0 = - \frac{f(r_0)}{f'(r_0)}$$

And so 

\begin{align}
f(r_1) 
&= f(r_0) - f'(c)\frac{f(r_0)}{f'(r_0)} \\
&= f(r_0)\rbr{1-\frac{f'(c)}{f'(r_0)}}
\end{align}

But since the function $f$ is convex in the open interval $(r',\infty)$, we know that $0 < f'(c) <f'(r_0)$.

Therefore 

$$0<f(r_1) < f(r_0)$$

which gives

$$r < r_1 < r_0$$

and finally

\begin{equation}\label{r1ineq}
r' < r < r_1
\end{equation}
\end{itemize}
And so $r_1>r$ in both cases, then Newton's method ensures $\lim\limits_{n\to\infty} r_n = r$, as required.
\end{pf}
\end{lemma}


\begin{definition}
The function $f$ takes on the average value between points $x_1$ and $x_2$, $f_{\text{avg}}$  given by the formula 
\begin{equation} \label{eq:avgvalue}
    f_{\text{avg}} = \frac1{x_2-x_1} \int_{x_1}^{x_2} f(x)dx
\end{equation}
\end{definition}

\begin{prop}\label{thm:avgparam}
The average value of the sequence $a_0,\dots,a_N$ defined by \eqref{eq:anper} can be reasonably estimated by our previous parameter $a$.

Similarly, the average value of the sequence $b_0,\dots,b_N$ defined by \eqref{eq:bnper} can be reasonably estimated by our previous parameter $b$.

\begin{pf}
Let $a(x)$ be the continuous extension of $a_0,a_1,\dots,a_N$, i.e. 

\begin{equation} \label{eq:axfn}
    a(x)=a\rbr{1+c_1\rbr{\sin\rbr{\frac{2\pi}{p_1}\rbr{x-n_1}}}},\quad 0\leq x \leq N
\end{equation}

Then, for $x_1=0$ and $x_2=N$

$\begin{aligned}
a_{\text{avg}} 
&= \frac{1}{N-0}\int_0^N a(x)dx \\
&= \frac{1}{N}\int_0^N a\rbr{1+c_1\rbr{\sin\rbr{\frac{2\pi}{p_1}\rbr{x-n_1}}}}dx\\
&= \frac1N \int_0^N a dx +  \frac{ac_1}{N} \int_0^N \sin\rbr{\frac{2\pi}{p_1}\rbr{x-n_1}} dx
\end{aligned}$

Then, using  $\sin(A-B) = \sin(A)\cos(B) - \cos(A)\sin(B)$

$\begin{aligned}
a_{\text{avg}} 
&= \frac1N ax\Big|_0^N  + \frac{ac_1}{N} \int_0^N \rbr{\sin\rbr{\frac{2\pi x}{p_1}}\cos\rbr{\frac{2\pi n_1}{p_1}} - \cos\rbr{\frac{2\pi x}{p_1}}\sin\rbr{\frac{2\pi n_1}{p_1}}}dx\\
&= \frac{aN}{N}  
+ \frac{ac_1}{N} \cos\rbr{\frac{2\pi n_1}{p_1}} \int_0^N \sin\rbr{\frac{2\pi x}{p_1}}dx 
-   \frac{ac_1}{N} \sin\rbr{\frac{2\pi n_1}{p_1}} \int_0^N \cos\rbr{\frac{2\pi x}{p_1}}dx\\
&= a 
+  \frac{ac_1}{N} \cos\rbr{\frac{2\pi n_1}{p_1}}  \cdot \rbr{- \frac{p_1}{2\pi}  \cos\rbr{\frac{2\pi x}{p_1}}}\Big|_0^N 
-   \frac{ac_1}{N} \sin\rbr{\frac{2\pi n_1}{p_1}}  \cdot \rbr{ \frac{p_1}{2\pi}  \sin\rbr{\frac{2\pi x}{p_1}}}\Big|_0^N \\
&= a 
-\frac{ac_1}{N} \cos\rbr{\frac{2\pi n_1}{p_1}} \frac{p_1}{2\pi} \cos\rbr{\frac{2\pi N}{p_1}} 
+\frac{ac_1}{N}\cos\rbr{\frac{2\pi n_1}{p_1}} \frac{p_1}{2\pi}  \cdot 1
-\frac{ac_1}{N} \sin\rbr{\frac{2\pi n_1}{p_1}} \frac{p_1}{2\pi}  \sin\rbr{\frac{2\pi N}{p_1}}
+0\\
&= a 
-\frac{ac_1 p_1}{2\pi N} \rbr{\cos\rbr{\frac{2\pi n_1}{p_1}}\cos\rbr{\frac{2\pi N}{p_1}} 
+\cos\rbr{\frac{2\pi n_1}{p_1}} 
-\sin\rbr{\frac{2\pi n_1}{p_1}}  \sin\rbr{\frac{2\pi N}{p_1}}}
\end{aligned}$


We use a similar identity $\cos(A+B) = \cos(A)\cos(B) - \sin(A)\sin(B)$ to get

$\begin{aligned}
a_{\text{avg}} 
&= a 
-\frac{ac_1 p_1}{2\pi N} \rbr{\cos\rbr{\frac{2\pi \rbr{N+n_1}}{p_1}}
+\cos\rbr{\frac{2\pi n_1}{p_1}}}
\end{aligned}$

Since $\cos(\cdot)$ is bounded between $\pm 1$ and

$c_1 p_1<0.2\cdot 8 = 1.6$ and $N>50$ usually, we see that

$$\left|\frac{c_1 p_1}{2\pi N} \rbr{\cos\rbr{\frac{2\pi \rbr{N+n_1}}{p_1}}
+\cos\rbr{\frac{2\pi n_1}{p_1}}}\right| < \frac{1.6}{300} \cdot \rbr{1+1}=\frac{4}{375}$$


So the average value of $a(x)$ is approximately within $4/375 \approx 1.07\%$ of the value of $a$.

Therefore $a$ is a reasonable estimate for the average $a_{\text{avg}}$ of the periodic parameter $a(x)$.

Similarly, exchanging $c_1,n_1,p_1$ for $c_2,n_2,p_2$ we have a definition for $b(x)$ and hence $b$ is a reasonable estimate for the average $b_{\text{avg}}$.

\end{pf}
\end{prop}

